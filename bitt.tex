\documentclass[12pt,a4paper]{article}
\usepackage{amsmath}
\usepackage{amsfonts}
\usepackage{amssymb}
\usepackage{amsthm}

\theoremstyle{definition}
\newtheorem{defn}{Definition}[section]
\newtheorem{defns}[defn]{Definitions}
\newtheorem{example}[defn]{Example}

\newtheorem{thm}{Theorem}[section]
\begin{document}
\author{Karim Osman}
\title{Brief Introduction to Topology \\ (working paper)}
\maketitle
\tableofcontents
\section{Introduction}
This short book started off as a series of notes on the fundamentals of
topology, a branch in mathematics that studies properties preserved under
continuous deformations of objects. As the number of notes grew, I felt that a
more organized set of writings would not only promote my understanding of the
subject, but also serve as a reference for people that wish to learn the basics
of topology.

Although, this book aims to be as complete as possible, I strongly recommend
that the reader follows some other introductory level publication on topology,
such as Bert Mendelson's Introduction to Topology: Third Edition published by
Dover Publications, and use the theorems and examples as a reference of how
profound is one's understanding of the subject; if you do not understand this
book, you do not understand topology. This said, the emphasize is strongly on
the presented examples and the remaining writings, such as definitions and
theorems, are meant to support understanding the examples.
\section{Basic Set Theory}
\subsection{Sets}
\begin{defns}
A set is a collection of zero or more objects (e.g. numbers, functions,
vectors, etc.), which are known as elements of a set.

We write $x \in A$ to denote that $x$ is an element of set $A$. If $x$ is not
an element of $A$, we write $x \notin A$.

If set $A$ is well-defined, then it is known whether an object is or is not an
element of set $A$.

A set is finite if the cardinality, the number of elements, is finite. The
cardinality of set $A$ is denoted by $\#A$. A set that is not finite is
infinite.
\end{defns}

\begin{defn}
The empty set is denoted by $\emptyset$ and it is finite. The empty set has no
elements and its cardinality is zero. If a set has one or more elements, it is
called $nonempty$.
\end{defn}

Sets are described by enclosing its elements in braces, where the right side of
the colon defines the condition for an object to be an element of a set:  $\{ x
: ...  \}$. A condition such as $x \in \Re$ can be placed on the left side of
the colon.

\begin{example}
The set $\{ x \in \Re : x < 2 \}$ is equal to $\{ x : x \in \Re \text{ and } x
< 2 \}$, and defines a collection of real numbers, in which all elements are
smaller than 2.
\end{example}

\begin{example}
Let $A = \{1, 2, 3, ... \}$. The cardinality of $A$ is undefined, therefore it
is infinite. However, $A$ is countable as each element can be associated with a
natural number.
\end{example}

\begin{defn}
A set $A$ is said to be a subset of $B$ and is denoted by $A \subset B$, if
each element in set $A$ is also an element of set $B$.
\end{defn}

Obviously, $A \subset B$ does not imply that $B \subset A$, however, if the
latter holds, the sets are equal. Note that it is important not to confuse
$\in$ and $\subset$. 

\begin{defn}
If $A$ and $B$ are sets, then $A = B$ if and only if $A \subset B$ and $B
\subset A$.
\end{defn}

By the definition of equality of sets, sets $A = \{1, 0\}$ and $B = \{1, 0, 1,
1\}$ are equal as each element of $A$ is also an element of $B$, and each
element of $B$ is also an element of $A$. In other words, repetition of
elements in a definition of a set does not change the set.

\begin{defns} 
A set $\{a, b\}$ is an unordered pair, whereas $(a, b)$ is an ordered pair.

Ordered pairs $(a, b)$ and $(c, d)$ are equal if and only if $a = c$ and $b =
d$.
\end{defns}

\subsection{Basic Operations on Sets}
\begin{defns}
Let $A$ and $B$ be sets. If $S = \{ x : x \in A \text{ or } x \in B \}$, then
$S$ is a union of $A$ and $B$, and we can write $S = A \cup B$.

Let $A_i$ be an indexed family of sets, where $i \in I$. If $S = \{ x : x \in
A_i \text{ for any } i \in I\}$, then $S$ is a union of sets $A_i$ for all $i
\in I$, and we can write $S = \bigcup_{i \in I} A_i$.  
\end{defns}

\begin{defns}
Let $A$ and $B$ be sets. If $S = \{ x : x \in A \text{ and } x \in B \}$, then
$S$ is an intersection of $A$ and $B$, and we can write $S = A \cap B$.

Let $A_i$ be an indexed family of sets, where $i \in I$. If $S = \{ x : x \in
A_i \text{ for all } i \in I\}$, then $S$ is an intersection of sets $A_i$ for
all $i \in I$, and we can write $S = \bigcap_{i \in I} A_i$.  
\end{defns}

\begin{defn}
Let $X$ be a set and $A \subset X$. The complement of $A$ in $X$ is denoted by
$A^c = X \backslash A$ or $\complement A = X \backslash A$. 
\end{defn}

\begin{example}
Let $X$ be a set and $A \subset X$. As $A \subset X$, then $\complement A \in
X$, while $\complement A \cap A = \emptyset$.
\end{example}

\begin{example}
Let $X = \{ x : x \in \mathbb{N} \}$. If $A = \{ x : x \in X \text{ and } x
\text{ is even} \}$, then $\complement A$ contains only odd natural numbers.
\end{example}

The following two theorems are known as the De Morgan's laws. These theorems
relate three basic operations on sets to each other: unions, intersections, and
complements.

\begin{thm}
If $X$ is a set and $A_i \subset X$ for all $i \in I$, then $\complement
\bigcup_{i \in I} A_i = \bigcap_{i \in I} \complement A_i$.
\begin{proof}
$x \notin \complement \bigcup_{i \in I} A_i \Leftrightarrow x \in \bigcup_{i
\in I} A_i \Leftrightarrow x \notin \bigcup_{i \in I} \complement A_i
\Leftrightarrow x \notin \bigcap_{i \in I} \complement A_i \Rightarrow
\complement \bigcup_{i \in I} A_i = \bigcap_{i \in I} \complement A_i$
\end{proof}
\end{thm}

\begin{thm}
If $X$ is a set and $A_i \subset X$ for all $i \in I$, then $\complement
\bigcap_{i \in I} A_i = \bigcup_{i \in I} \complement A_i$.
\begin{proof}
$x \notin \complement \bigcap_{i \in I} A_i \Leftrightarrow x \in \bigcap_{i
\in I} A_i \Leftrightarrow x \in \bigcup_{i \in I} A_i \Leftrightarrow x \notin
\bigcup_{i \in I} \complement A_i \Rightarrow \complement \bigcap_{i \in I} A_i
= \bigcup_{i \in I} \complement A_i$
\end{proof}
\end{thm}

\begin{defn}
Let $A$ and $B$ be sets. The Cartesian product of $A$ and $B$ is denoted by $A
\times B$, and the product, is a set of ordered pairs: $A \times B = \{ (a, b)
: a \in A \text{ and } b \in B \}$. Consider the Cartesian product $\Re \times
\Re$, then the set of ordered pairs are coordinates in $\Re^2$.
\end{defn}

\subsection{Ordered Sets}
The ordered sets that are of our interest are partially ordered sets and
totally ordered sets.
\begin{defn}
Let $A$ be a partially ordered set and $x$, $y$, and $z$ are elements of $A$.
An order on set $A$ is a relation, denoted by $\leq$, with the following three
properties:
\begin{enumerate}
\item Reflexivity: $x \leq x$ 
\item Antisymmetry: if $x \leq y$ and $y \leq x$, then $x = y$
\item Transitivity: if $x \leq y$ and $y \leq z$, then $x \leq y$
\end{enumerate}
\end{defn}
\begin{defn}
Let $A$ be a totally ordered set and $x$, $y$, and $z$ are elements of $A$.
An order on set $A$ is a relation, denoted by $\leq$, with the following three
properties:
\begin{enumerate}
\item Antisymmetry: if $x \leq y$ and $y \leq x$, then $x = y$
\item Transitivity: if $x \leq y$ and $y \leq z$, then $x \leq y$
\item Totality: $x \leq y$ or $y \leq x$
\end{enumerate}
\end{defn}
The relation among elements in a set form an order in which the elements are in
the set, and depending on the properties that the relation among elements in a
set satisfy, a set is either ordered or unordered.

\begin{defns}
Let $X$ be a partially ordered set and $A \subset X$. The elements of $X$ that
are greater than or equal to every element in $A$ are known as the upper bounds
of set $A$.

Let $X$ be a partially ordered set and $A \subset X$. The elements of $X$ that
are are smaller than or equal to every element in set $A$ are known as the
lower bounds of set $A$. 
\end{defns}

\begin{defn}
Let $X$ be a partially ordered set, $A \subset X$, and $S$ is the set of upper
bounds of $A$. The smallest upper bound of $A = \min S$, is the supremum of
$A$, which is denoted by $\sup A$.
\end{defn}

Respectively, the properties of the greatest lower bound, infimum, for set $A$
is obtained by reversing the order of $A$ by multiplying it by $-1$.

\begin{example}
If $S = \{ x \in \Re : x < 1 \}$, then $\sup S = 1$ and $\max S$, $\min S$, as
well as $\inf S$ do not exist.
\end{example}

\begin{example}
Let $A \subset \Re$ and $A \neq \emptyset$ and assume that $A$ has an upper
bound. Let $S$ be a nonempty set of upper bounds of $A$, that is, $S = \{ a \in
\Re : x \leq a \text{ for all } x \in A \}$. As $S$ has a lower bound and per
the completeness of real numbers, there exists a smallest number in $S$, which
is the supremum of $A$, or more formally, $\sup A = \min S \in \Re$.  Let's
show that $\sup A$ is uniquely defined using the properties of partially
ordered sets. Note that $\Re$ is a partially ordered set (and totally ordered
set).
\begin{proof}
Let $x = \min S$ and $y = \min S$. The relation on $S$ is antisymmetric and
reflexive: $x \in S$ then $x \leq y$, and $y \in S$ then $y \leq x \Rightarrow
x = y$.
\end{proof}
\end{example}

\begin{example}
Let's show that if $\max A$ exists then $\sup A = \max A$.
\begin{proof}
The relation on $\Re$ is antisymmetric, reflexive, and transitive: $\max A \in
S$ and $\sup A = \min S$, then $\sup A \leq \max A$, and $\max A \in A$ and
$\sup A \in S$, then $\max A \leq \sup A \Rightarrow \max A = \sup A$.
\end{proof}
\end{example}

\begin{example}
Let's show that when $\epsilon > 0$ then there exists such $x \in A$ that
satisfies $x > \sup A - \epsilon$.
\begin{proof}
Let's proof by contradiction. That is, $x \leq \sup A - \epsilon$. Recall that
$S = \{ a \in \Re : x \leq a \text{ for all } x \in A \}$, then $\sup A -
\epsilon \in S$, because $x \leq a$ and we choose $a = \sup A - \epsilon$. As
$\sup A - \epsilon \in S$, then $\min S \leq \sup A - \epsilon \Leftrightarrow
\min S \leq \min S - \epsilon < \min S$, which does not hold as $\min S$ is
uniquely defined.
\end{proof}
\end{example}

\subsection{Functions}
\begin{defns}
Let $X$ and $Y$ be sets. The mapping (or function) $f$ from $X$ into $Y$ maps
each $x \in X$ to exactly one element in $Y$, and we can write $f: X
\rightarrow Y$. Sets $X$ and $Y$ are known as domain and range, respectively.
\end{defns}
One should not confuse function $f$ with an image of $f$ denoted by $f(x)$,
where $x \in X$.
\begin{defn}
Let $f$ be a function. If $f(x) = f(x')$ holds only when $x = x'$, then $f$ is
injective.
\end{defn}

\begin{defn}
Let $f: X \rightarrow Y$. If for all $y \in Y$ there exists such $x \in X$ that
$y = f(x)$, then $f$ is surjective.
\end{defn}

\begin{defn}
If function $f$ is injective and surjective, then $f$ is bijective.
\end{defn}

\begin{defn}
Let $f: X \rightarrow Y$. The inverse function for $f$, denoted by $f^{-1}$, is
a mapping from $Y$ onto $X$, or $f^{-1}: Y \rightarrow X$.
\end{defn}

\begin{defn}
Let $f: X \rightarrow Y$ and $g: Y \rightarrow Z$. The composite function of
$f$ and $g$ is denoted by $f \circ g$, and the image of the composite function
$f \circ g$ is denoted by $(f \circ g)(x) = f(g(x))$.
\end{defn}


\begin{thm}
Let $f: X \rightarrow Y$. An inverse function $f^{-1}$ exists if and only if
$f$ is a bijection. 
\begin{proof}
Let's proof by contradiction. Let $f: X \rightarrow Y$.

\begin{enumerate}
\item Injective: Assume that $f^{-1}: Y \rightarrow X$ exists. If $x \neq x'$
and $y = f(x) = f(x')$, then $f^{-1}(y) = x'$ or $f^{-1}(y) = x$, which is in
contradiction with the definition of a function, and therefore $f^{-1}$ does
not exist.

\item Surjective: Assume that $f^{-1}: Y \rightarrow X$ exists. If $y \in Y$
and $y = f(x)$ does not hold for any $x \in X$, then $f^{-1}(y) \notin X$,
which is in contradiction with the definition of a function, and therefore
$f^{-1}$ does not exist.
\end{enumerate}
From (1) and (2) it follows that $f^{-1}$ exists if and only if $f$ is
bijective.
\end{proof}
\end{thm}

\begin{defn}
A function that maps each element to itself is an identity function, or $f: X
\rightarrow X$ for all $x \in X$, is usually denoted by $id_X$. Let $g: X
\rightarrow Y$ be bijective, then we write $g \circ g^{-1} = id_Y$ and $g^{-1}
\circ g = id_X$.
\end{defn}

\begin{thm}
If $f: X \rightarrow Y$, $g: Y \rightarrow X$, $f \circ g = id_Y$ and $g \circ
y = id_X$, then $f$ and $g$ are bijections and $f = g^{-1}$.
\begin{proof}
Let's show that $f$ is injective and surjective, and therefore bijective (the
proof that $g$ is bijective is similar):
\begin{enumerate}
\item Injectivity: Let $x_1 \in X$, $x_2 \in X$, and $x_1 \neq x_2$. By the
definition of a function (for each $y \in Y$ there exists exactly one $x \in
X$), if $g(f(x_1)) \neq g(f(x_2))$, then $f(x_1) \neq f(x_2)$.
\item Surjectivity: Let $y \in Y$. As $g(y) \in X$ and $f(g(y)) \in Y$, then
for all $x \in X$ there exists $f(x) = y$.
\end{enumerate}
We know that $f^{-1} \circ f = id_X$ and $f \circ g = id_Y$, so $g = id_X \circ
g \Leftrightarrow g = f^{-1} \circ f \circ g \Leftrightarrow g = f^{-1} \circ
id_Y \Leftrightarrow g = f^{-1}$.
\end{proof}
\end{thm}

\begin{thm}
If $f: X \rightarrow Y$ is a bijection, then $f^{-1}: Y \rightarrow X$ is a
bijection, and $(f^{-1})^{-1} = f$.
\begin{proof}
Let's show that $f^{-1}$ is injective and surjective, and therefore bijective:
\begin{enumerate}
\item Injectivity: Let $y_1 \in Y$, $y_2 \in Y$, and $y_1 \neq y_2$. If
$f^{-1}$ is not injective, then by injectivity of $f$, $f^{-1}(y_1) =
f^{-1}(y_2) \Leftrightarrow f(f^{-1}(y_1)) = f(f^{-1}(y_2)) \Leftrightarrow y_1
= y_2$, which does not hold.
\item Surjectivity: If $f^{-1}$ is not surjective, then for $x \in X$ there
does not exist such $y \in Y$ that $x = f^{-1}(y) \Leftrightarrow f(x) =
f(f^{-1}(y)) \Leftrightarrow y = f(x)$, which does not hold, as $f$ is
surjective.
\end{enumerate}
From (1) and (2) it follows that $f^{-1}$ is a bijection. 

Let $g = f^{-1}$. As $id_X = g \circ g^{-1}$ and $id_Y = f \circ f^{-1} = f
\circ g$, then $f = id_Y \circ f = f \circ g \circ f = f \circ id_X = f \circ
f^{-1} \circ (f^{-1})^{-1} = id_Y \circ (f^{-1})^{-1} = (f^{-1})^{-1}$.
\end{proof}
\end{thm}
\section{Inner Product Spaces and Normed Spaces}
\subsection{Euclidean Space and Vector Space}
\begin{defns}
Let $x$ and $y$ be vectors in an Euclidean space $\Re^n$. The inner product (or
the dot product) of $x$ and $y$ is $x \cdot y = \sum^n_{i=1} x_i y_i$. The norm
of $x$ is $ ||x|| = \sqrt{x \cdot x}$.
\end{defns}
Choosing these two basic properties of an Euclidean space as axioms, the
concept of an inner product space and normed vector space can be defined.
\begin{defn}
A vector space $E$ is a set of vectors given that:
\begin{enumerate}
\item for each pair of vectors $x \in E$ and $y \in E$, the sum $x + y \in E$
\item for each vector $x$ and real number $a$, the product $a x \in E$
\end{enumerate}
which are subject to the following axioms for all vectors $x, y, z$ and $a, b
\in \Re$:
\begin{enumerate}
\item Associativity of addition: $(x + y) + z = x + (y + z)$.
\item Commutativity of addition: $(x + y) = (y + x)$.
\item Identity element of addition: there exists an element $0 \in E$ (zero
vector) such that $0 + x = x$.
\item Inverse element of addition: for each $x \in E$ there exists $-x \in E$
such that $x + (-x) = 0$.
\item Distributivity of scalar multiplication: $a (x + y) = ax + ay$, $(a + b)x
= ax + bx$.
\item Comptatibility of scalar multiplication: $(ab)x = a(bx)$.
\item Identity element of scalar multiplication: $1 x = x$.
\end{enumerate}
\end{defn}
\subsection{Inner Product Space}
\begin{defns}
Let $E$ be a vector space. Image $(x, y) \rightarrow x \cdot y: E \times E
\rightarrow R$ is an inner product in $E$ if for all $x,y,z \in E$ and $a \in
R$:
\begin{enumerate}
\item $x \cdot y = y \cdot x$
\item $(ax) \cdot y = a (x \cdot y)$
\item $(x + y) \cdot z = x \cdot z + y \cdot z$
\item $x \cdot x \geq 0$
\item $x \cdot x = 0 \Leftrightarrow x = 0$
\end{enumerate}
A vector space for which an inner product is defined is also an inner product
space.
\end{defns}
\begin{example}
Let's proof that $\Re^n$ is an inner product space by showing that each of the
aformentioned properties hold for $\Re^n$.
\begin{proof}
Let $x = (x_1,...,x_n), y = (y_1,...,y_n), z = (z_1,...,z_n) \in \Re^n$ and $a
\in \Re$.
\begin{enumerate}
\item As $x_i y_i = y_i x_i$: $x \cdot y = \sum_{i = 1}^{n}(x_i y_i) = \sum_{i
= 1}^n (y_i x_i) = y \cdot x$
\item As $(a x_1,...,a x_n) = ax$: $(ax) \cdot y = \sum_{i=1}^n (a x_i) y_i = a
\sum_{i=1}^n (x_i y_i) = a (x \cdot y)$
\item As $x + y = (x_1 + y_1,...,x_n + y_n)$: $(x + y) \cdot z = \sum_{i=1}^n
(x_i + y_i) z_i = \sum_{i=1}^n (x_i z_i + y_i z_i) = \sum_{i=1}^n x_i z_i +
\sum_{i=1}^n y_i z_i = x \cdot z + y \cdot z$
\item As $x_i^2 = x_i x_i \geq 0$ for all $i \in I = \{1,...,n\}$: $x \cdot x =
\sum_{i=1}^n x_i x_i \geq 0$
\item As $x_i x_i = 0 \Rightarrow x_i = 0$ for all $i \in I = \{1,...,n\}$: $x
\cdot x = \sum_{i=1}^n x_i x_i \Rightarrow \sum_{i=1}^n x_i = x = 0$
\end{enumerate}
\end{proof}
\end{example}
\subsection{Normed Vector Space}

\section{Metric Spaces}

[ TODO: Write it. ]

\section{Open Sets}

[ TODO: Write it. ]

\section{Closed Sets}

[ TODO: Write it. ]

\section{Continuous Functions}

[ TODO: Write it. ]

\section{Subspace Topology}

[ TODO: Write it. ]

\begin{thebibliography}{9}
\bibitem{schecter97}
  Eric Schecter,
  \emph{Handbook of Analysis and Its Foundations}.
  Academic Press, New York,
  1997.
\bibitem{gamelin99}
  Theodore W. Gamelin, Robert Everist Greene,
  \emph{Introduction to Topology}.
  Dover Publications, New York,
  Second Edition,
  1999.
\bibitem{mendelsonXX}
  Bert Medelson,
  \emph{Introduction to Topology}.
  Dover Publications, New York,
  Third Edition,
  1990.
\bibitem{rudin76}
  Walter Rudin,
  \emph{Principles of Mathematical Analysis}
  McGraw-Hill Science/Engineering/Math, 
  Third Edition,
  1976.
\end{thebibliography}
\end{document}
